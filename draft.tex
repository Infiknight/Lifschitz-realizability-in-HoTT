\documentclass[12pt]{report}

\usepackage{tikz-cd}
%\usepackage{answers}
\usepackage{biblatex}
\addbibresource{bibliography.bib}
\usepackage{setspace}
%\usepackage{graphicx}
\usepackage{enumitem}
%\usepackage{tikz-cd}
%\usepackage{todonotes}
%\usepackage{multicol}
%\usepackage{mathrsfs}
%\usepackage[margin=1in]{geometry} 
\usepackage{hyperref}% http://ctan.org/pkg/hyperref
\usepackage{ stmaryrd }
\usepackage{amsmath}
\usepackage{cleveref}% http://ctan.org/pkg/cleveref
\usepackage{amsthm}
\usepackage{amssymb}
\usepackage{ textcomp }
%\usepackage{ wasysym }
\usepackage{bookmark}
%\usepackage{accents}
\usepackage[parfill]{parskip}

 
\newcommand{\N}{\mathbb{N}}
\newcommand{\Z}{\mathbb{Z}}
\newcommand{\R}{\mathbb{R}}
\newcommand{\nin}{\not \in}

\DeclareMathOperator{\sech}{sech}
\DeclareMathOperator{\csch}{csch}
 
\newenvironment{theorem}[2][Theorem]{\begin{trivlist}
\item[\hskip \labelsep {\bfseries #1}\hskip \labelsep {\bfseries #2.}]}{\end{trivlist}}
\newenvironment{definition}[2][Definition]{\begin{trivlist}
\item[\hskip \labelsep {\bfseries #1}\hskip \labelsep {\bfseries #2.}]}{\end{trivlist}}
\newenvironment{proposition}[2][Proposition]{\begin{trivlist}
\item[\hskip \labelsep {\bfseries #1}\hskip \labelsep {\bfseries #2.}]}{\end{trivlist}}
\newenvironment{lemma}[2][Lemma]{\begin{trivlist}
\item[\hskip \labelsep {\bfseries #1}\hskip \labelsep {\bfseries #2.}]}{\end{trivlist}}
\newenvironment{exercise}[2][Exercise]{\begin{trivlist}
\item[\hskip \labelsep {\bfseries #1}\hskip \labelsep {\bfseries #2.}]}{\end{trivlist}}
\newenvironment{solution}[2][Solution]{\begin{trivlist}
\item[\hskip \labelsep {\bfseries #1}]}{\end{trivlist}}
\newenvironment{problem}[2][Problem]{\begin{trivlist}
\item[\hskip \labelsep {\bfseries #1}\hskip \labelsep {\bfseries #2.}]}{\end{trivlist}}
\newenvironment{question}[2][Question]{\begin{trivlist}
\item[\hskip \labelsep {\bfseries #1}\hskip \labelsep {\bfseries #2.}]}{\end{trivlist}}
\newenvironment{corollary}[2][Corollary]{\begin{trivlist}
\item[\hskip \labelsep {\bfseries #1}\hskip \labelsep {\bfseries #2.}]}{\end{trivlist}}

\usepackage{xcolor}
\newcommand\todo[1]{\textcolor{red}{#1}}
 
%\newenvironment{bprooftree}
 % {\leavevmode\hbox\bgroup}
  %{\DisplayProof\egroup}

\newtheorem{thm}{Theorem}[section]
\newtheorem{defn}[thm]{Definition}
\newtheorem{lem}[thm]{Lemma}
\newtheorem{cor}[thm]{Corollary}
\newtheorem{prop}[thm]{Proposition}
\newtheorem{axiom}[thm]{Axiom}
\newtheorem{rem}[thm]{Remark}
\newtheorem{ill}[thm]{Illustration}
\crefname{thm}{Lemma}{Lemmas}



\begin{document}
 %\pagenumbering{gobble}
% --------------------------------------------------------------
%                         Start here
% --------------------------------------------------------------
 
\title{Draft}%replace with the appropriate homework number
\author{Dimitrios Koutsoulis\\ %replace with your name
11838639} %if necessary, replace with your course title
 
\begin{titlepage}
\hypersetup{pageanchor=false}
\maketitle
\thispagestyle{empty}
\end{titlepage}
\hypersetup{pageanchor=true}
%Below is an example of the problem environment
\chapter{Type Theory}
\section{Introduction}
Type theory is a formal language and deductive system, that is self sufficient in the sense that it need not be formulated as a collection of axioms on top of some other formal system like First Order Logic, instead its deductive system can be built on top of its own formal language. 

Central to Type Theory is the notion of \textit{Type}. Every term $a$ in Type Theory we come across, must lie in some type $A$, which we denote as $a : A$. 
To avoid impredicativity we assume a countable hierarchy of universes $\mathcal{U}_0,\mathcal{U}_1,\mathcal{U}_2,\ldots$ which is cummulative i.e. every universe includes all previous universes and their types. 
While working in type theory we usually simplify our view of this hierarchy and pick some $\mathcal{U}$, of arbitrary index that we do not specify, to use as our workspace. 
%Note that the relation $:$ is transitive, so $a : A$ and $A : B$ imply $a : B$. 
%This induces the cummulative hierarchy shape of the universe of types $\mathcal{U}$, wherein every level of the universe includes all lower levels and their types.

For the deductive part of Type Theory, we interpret propositions as types. Proving proposition $P$ amounts to providing some inhabitant $p : P$. 

\section{Type Construction Operations}
Let's have a look at some important type constructions.
\begin{itemize}
	\item Given types $A, B : \mathcal{U}$ we can define the type $A \rightarrow B$ of \textbf{functions} from $A$ to $B$. 
		We can use $\lambda$-abstraction to construct elements of this type. $\lambda x. \Phi$ lies in $A \rightarrow B$ iff for $a : A$ we have $\Phi[a/x] : B$. 
		For $f : A \rightarrow B$ and $a : A$ we have that the application of $f$ on $a$, denoted as $f(a)$ or $f\;a$, lies in $B$.
	\item Given some type $A : \mathcal{U}$ and a family of types $B$ over $A$, $B : A \rightarrow \mathcal{U}$, we have the type of \textbf{dependent functions} or \textbf{dependent products} (which should not be viewed as the dependent version of the product type defined bellow) $$\prod_{a : A} B(a)$$
	where for $f : \prod_{a : A} B(a)$ and $x : A$ we have $f(x) : B(x)$. 
	As in the case of non-dependent functions, we can use lambda abstraction to construct elements of a dependently-typed function type. 
	That way, $\lambda x.\Phi$ lies in $\prod_{a:A}B(a)$ iff $B$ is of the form $\lambda y.\Psi$ and for all $a:A$ we have $\Phi[a/x] : \Psi[a/y]$. 
	\item Given $A, B : \mathcal{U}$ we can define the \textbf{product} type $A \times B : \mathcal{U}$. 
	For $a : A$ and $b : B$ we have the pair $(a,b) : A \times B$. 
	We also have the projection functions 
	$$\mathtt{pr}_1 : A \times B \rightarrow A : (a,b) \mapsto a$$
	$$\mathtt{pr}_2 : A \times B \rightarrow B : (a,b) \mapsto b$$
	We also have the following induction principle
	$$\mathtt{ind}_{A\times B} : \prod_{C: A\times B \rightarrow \mathcal{U}}\big( \prod_{a:A}\prod_{b:B} C((a,b)) \big) \rightarrow \prod_{x: A\times B} C(x)$$
	$$\mathtt{ind}_{A\times B}(C, f, (a,b)) : \equiv f(a)(b)$$
	So given two functions $f: A\rightarrow C$ and $g : \rightarrow C$ we can construct an $h : A\times B \rightarrow C$ such that $h((a,b)) \equiv \big(f(a), g(b)\big)$ for all $a:A,\; b:B$.
	\item Given $A : \mathcal{U}$ and family of types $B$ over $A$, $B : A \rightarrow \mathcal{U}$, we can define the \textbf{dependent pair} type (the dependent version of the product type)
	$$\sum_{a : A} B(a)$$
	Given $x : A$ and $b : B(x)$ we can construct the pair $(x,b) : \sum_{a : A} B(a)$. 
	We have two projection functions, similar to the case of the product type.
	$$\mathtt{pr}_1 : \sum_{a:A}B(a) \rightarrow A : (a,b) \mapsto a$$
	$$\mathtt{pr}_2 : \prod_{x : \sum_{a:A}B(a)}B(\mathtt{pr}_1\; x) : (a,b) \mapsto b$$
	The induction principle is the following
	$$\mathtt{ind}_{\sum_{a:A}B(a)} : \prod_{C : \sum_{a:A}B(a) \rightarrow \mathcal{U}}\Big( \prod_{a : A}\prod_{b : B(a)}C((a,b)) \Big)\prod_{x : \sum_{a:A}B(a)}C(x)$$
	$$\mathtt{ind}_{\sum_{a:A}B(a)}(C, f,(a,b) ) :\equiv f(a)(b)$$
	\item Given $A, B : \mathcal{U}$ we can construct the \textbf{coproduct} type $A + B$. 
	We can construct elements of $A + B$ using the functions
	$$\mathtt{inl} : A \rightarrow A+B$$
	$$\mathtt{inr} : B \rightarrow A+B$$
	One can guess the induction principle
	$$\mathtt{ind}_{A+B} : \prod_{C : A+B \rightarrow U} \Big(\prod_{a : A}C(\mathtt{inl}\;a)\Big) \rightarrow \Big(\prod_{b : B}C(\mathtt{inr}\;b)\Big) \rightarrow \prod_{x : A+ B}C(x)$$
	$$\mathtt{ind}_{A+B}(C, f_A, f_B, \mathtt{inl}\; a) :\equiv f_A\; a$$
	$$\mathtt{ind}_{A+B}(C, f_A, f_B, \mathtt{inr}\; b) :\equiv f_B\; b$$
	\item Given $x, y : A$ we have the \textbf{identity type} $x =_A y$. 
	An element of this type amounts to a proof that $x$ and $y$ are equal. 
	Say $x$ and $y$ are judgmentally equal. 
	This is captured by the element $\mathtt{idp}_x : x =_A y$. 
	The relevant induction principle describes how we can use elements of an identity type
	$$\mathtt{ind}_{=_{A}} : \prod_{C : \prod_{x,y : A} (x =_A y) \rightarrow \mathcal{U}} \Big( \prod_{x : A}C(x,x,\mathtt{idp}_x) \Big) \rightarrow \prod_{x,y : A} \prod_{(p : x =_A y)} C(x,y,p)$$
	The relevant computation gives us the judgmental (definitional) equality 
	$$\mathtt{ind}_{=_A} (C,c,x,x,\mathtt{idp_x}) \equiv c\; x$$
	We can concatenate those paths whose domains and codomains allow for it. 
	Paths are equivalences. That is if $p : x = y$ is such a  path, we can provide its inverse $p^{-1}$ for which we have in turn a path between $p\cdot p^{-1}$ and $\mathtt{idp}_y$ and another one between $p^{-1}\cdot p$ and $\mathtt{idp}_x$. 
\end{itemize}
	\begin{defn}
	We call a type $A$ a \textbf{mere proposition} or simply a \textbf{proposition}, if for every $a,b : A$ we have $a = b$.
	\end{defn}
\begin{itemize}
	\item For every type $A : \mathcal{U}$ there exists its \textbf{propositional truncation} $\| A \|$. 
	We also have the truncation map $|\cdot| : A \rightarrow \lVert A \rVert$ so that for every element $a : A$ there exists $|a| : \| A\|$. 
	For every $x, y : \| A\|$ we have $x = y$. 

	Given mere proposition $B$ and $f : A \rightarrow B$, the recursion principle gives us $g : \|A\| \rightarrow B$ such that $g(|a|) \equiv f(a)$ for all $a : A$.
	This recursion principle will prove itself an indispensable tool in the sections to come. 
	Whenever, in the midst of a proof, the current goal $B$ is a mere proposition and we have access to some witness $a : \lVert A \rVert$, we are allowed by the recursion principle to assume that we have $a' : A$ and use it to construct a proof of $B$.
\end{itemize}
	\begin{defn}
	Let $f,g : \prod_{a:A}B(a)$ where $B :A \rightarrow \mathcal{U}$. 
	We call a function of the following type a \textbf{homotopy} between $f$ and $g$
	$$f\sim g :\equiv \prod_{a:A}f(a) = g(a)$$
	\end{defn}

	\begin{defn}
	We call a type $A$ a \textbf{set} if for every $a,b : A$ we have that $a =_A b$ is a mere proposition.
	\end{defn}

	\begin{defn}
	We call a type $A$ \textbf{contractible} if there exists $a : A$ such that for all $x : A$ it holds that $x = a$.\\
	Formally
	$$\mathtt{isContr}\; A :\equiv \sum_{x : A}\prod_{a :A}x=a$$
	\end{defn}

	\begin{defn}
	Given some map $f : A \rightarrow B$, a \textbf{fiber} of it over some point $y : B$ is
	$$\mathtt{fib}_f\; y :\equiv \sum_{x : A} \big(f(x) = y\big)$$
	\end{defn}

	\begin{defn}
	We say that a map $f : A\rightarrow B$ has \textbf{contractible fibers} if for every $b : B$, the type $\mathtt{fib}_f(b)$ is contractible.\\
	Formally, 
	$$ \mathtt{hasContrFibers}\; f : \equiv \prod_{b : B} \mathtt{isContr}(\mathtt{fib}_f\; b)$$
 	\end{defn}
 	One way to see this, albeit naive from a set-theoretic point of view, is that we require for every element of the codomain to have exactly one element of the domain mapped to it by $f$. 
 	Note that for any map $f$, the type $\mathtt{hasContrFibers(f)}$ is a mere proposition. 

 	We form a notion of \textbf{equivalence of types} based on maps with contractible fibers. 
 	Whenever we have some $f : A\rightarrow B$ with contractible fibers, we say that the types $A$ and $B$ are equivalent and write $A \simeq B$. 
 	To motivate this, note that we can form $g : B \rightarrow A$ so that $f\circ g \sim \mathtt{id}_B$ and $g\circ f \sim \mathtt{id}_A$. 
 	So we can use a witness of one of the types to get one of the other. 
 	Furthermore, the equivalence of types we introduced just now is an equivalence relation on $\mathcal{U}$. 
 	More exposition on equivalence can be found in chapter 4 of \cite{hottbook}.

 	We will now see some Type Theory variants that expand a bit upon what we've laid out in this section.

\section{Homotopy Type Theory}
We get the flavour of Homotopy Type Theory (HoTT) that interests us by assuming the axiom of Univalence.
\begin{lem}
We can define the following function
$$\mathtt{idtoequiv} : A=_\mathcal{U}B \rightarrow A\simeq B$$
\end{lem}
\begin{proof}
The definition of $\mathtt{idtoequiv}$ can be found in section 2.10 of \cite{hottbook}. 
\end{proof}

\begin{axiom}[Univalence]
$\mathtt{idtoequiv}$ has contractible fibers.
\end{axiom}
In HoTT we usually assume that the universe $\mathcal{U}$ that we are working in is univalent which means that for all $A,B: \mathcal{U}$
$$A=_\mathcal{U}B \simeq (A \simeq B)$$

\section{Extensional Type Theory}
Extensional Type Theory (ETT) is not consistent with the Homotopy Type Theory defined above. 
To get ETT we assume the axiom K.
\begin{axiom}[K]
All types are sets.
\end{axiom}

This is enough talk about variants of Type Theory.
\section{Logic}
Our informal deductions in Type Theory will be reminiscent of First Order Logic ones. 
To be able to use a similar verbiage, we will set down a handful of types, corresponding to the connectives that let us form well-formed formulas in FOL. 
These types need to be mere propositions, so that we can form non-constructive deductions. 
This approach is called `propositions as mere propositions' in \cite{hottbook}. 
In the following, $A$ and $B$ are mere propositions.
\begin{itemize}
\item When we talk of conjunction $A \wedge B$, we mean the product $A \times B$. 
\item We interpret $A \vee B$, as the truncation $\| A + B\|$.
\item We interpret $\forall a\in A,\; P(a)$, where $P(a)$ is a mere proposition for all $a \in A$, as $\prod_{a : A}P(a)$. 
\item We interpret $ \exists a \in A,\; P(a)$, where $P(a)$ is a mere proposition for all $a \in A$, as $\| \Sigma_{a : A} P(a) \|$
\end{itemize}
The use of the above notation is interchangeable in the sections to follow.


\chapter{Modalities}
%Under the view of types as $(\infty, 1)$-categories, functions between them are functors. 
%We can then view the universe $\mathcal{U}$ itslef as a category. 
\begin{defn}\label{modality_definition}
A \textbf{modality} is a function $\bigcirc : \mathcal{U} \rightarrow \mathcal{U}$ with the following properties.
\begin{enumerate}
	\item For every type $A$ we have a function $\eta_A^\bigcirc : A \rightarrow \bigcirc\; A$
	\item for every $A : \mathcal{U}$ and every type family $B : \bigcirc A \rightarrow \mathcal{U}$ we have a function
	$$\mathtt{ind}_\bigcirc : \Big( \prod_{a : A}\bigcirc (B (\eta_A^\bigcirc\; a)) \Big) \rightarrow \prod_{z : \bigcirc A} \bigcirc (B\; z)$$
	\item For every $f : \prod_{a : A} \bigcirc (B(\eta_A^\bigcirc\; a))$ and every $a : A$, there is a path $\mathtt{ind}_\bigcirc (f)(\eta_A^\bigcirc\; a) = f\; a$
	\item For all $z,z' : \bigcirc \; A$, the function $\eta_{z=z'}^\bigcirc : (z = z') \rightarrow \bigcirc (z = z')$ is an equivalence.

\end{enumerate}
\end{defn}
A modality $\bigcirc$ induces a $\Sigma$-closed reflective subuniverse.

\begin{defn}
Given modality $\bigcirc : \mathcal{U} \rightarrow \mathcal{U}$, a \textbf{reflective subuniverse} is a `subset' of $\mathcal{U}$ encoded by a family of h-propositions $P : \mathcal{U} \rightarrow \mathtt{Prop}$ such that the following conditions hold. 
\begin{itemize}
	\item For $A : \mathcal{U}$, we have $P(\bigcirc A)$.
	\item For $A : \mathcal{U}$ and $B$ such that $P(B)$, the function
	$$\lambda f.f \circ \eta_A^\bigcirc : (\bigcirc A \rightarrow B) \rightarrow (A \rightarrow B)$$
	is an equivalence.

\end{itemize}
The subuniverse is $\Sigma$\textbf{-closed} if for $X$ such that $P(X)$ and $Q : X \rightarrow \mathcal{U}$ such that $\prod_{x : X} P(Q(x))$, we have $P(\Sigma_{x : X} Q(x))$.
\end{defn}

\begin{thm}
Reflective subuniverses are closed under products. 
That is for subuniverse $P$ and $B : A \rightarrow \mathcal{U}$ such that $\prod_{a : A}P(B(a))$, we have that $P(\prod_{a : A} B(a))$. 
\end{thm}

\begin{proof}
For $a : A$, consider $\mathtt{ev}_a: (\prod_{a: A}B(a)) \rightarrow B(a)$ defined by $\mathtt{ev}_a (f) : \equiv f(x)$. 
Since $P(B(a))$, we have 
$$(\lambda f.f \circ \eta_{\prod_{a: A}B(a)}^\bigcirc)^{-1} (\mathtt{ev}_a) : \bigcirc\big(\prod_{a: A}B(a)\big) \rightarrow B(a)$$
We can now define the retraction of $\eta^\bigcirc_{\prod_{a : A} B(a)}$ by pattern matching as such:\\
For $z: (\prod_{a: A}B(a))$ and $a : A$ we have 
$$(\lambda f.f \circ \eta_{\prod_{a: A}B(a)}^\bigcirc)^{-1} (\mathtt{ev}_a) (z) : B(a)$$

\end{proof}

\begin{defn}\label{Bnull}
For $B : A \rightarrow \mathcal{U}$, we call $X$ $B$\textbf{-null} if the map 
$$\lambda x. \lambda b.x : X \rightarrow (B(a) \rightarrow X)$$
is an equivalence for all $a : A$.

\end{defn}

Nullification at a family of types is an example of a modality, as laid out in \cite{1706.07526}, where it is presented as a higher inductive type, complete with constructors and eliminators. 
To avoid listing all these trappings, we will instead look at the important properties $\mathcal{L}_B : \mathcal{U} \rightarrow \mathcal{U}$ should have to be considered a nullification operator. 

It must have all the data required for it to be a modality.\\
For all $X : \mathcal{U}$, $\mathcal{L}_B(X)$ must be $B$\textbf{-null}.

\chapter{Computability}
In this chapter we will give an overview of basic definitions and results of Recursion theory and formalize them in Type theory. 
The reader is expected to be somewhat familiar with Turing machines, as we will not be delving into their technical details. 
%Instead, we follow a similar approach to that of \cite{bridges_richman_1987} and assume an effective enumeration of 

\begin{defn}
A \textbf{partial function} is a function with a subset of the naturals as its domain.\\
Formally 
$$f : \Big(\sum_{n : \mathbb{N}} P\; n \Big) \rightarrow \mathbb{N}$$
where $P : \mathbb{N} \rightarrow \mathbf{Prop}$ is a family of propositions over $\mathbb{N}$, which works as the characteristic function of the domain of the partial function $f$.
\end{defn}

A standard result in Recursion theory is that there exists a Turing machine that enumerates all Turing Machines. 
We assume a fixed enumeration $T_1, T_2,\ldots $ that we refer to going forward. 

\begin{defn}\label{CT}
Church's thesis $(CT)$ is an axiom to be assumed, which states that every function $\mathbb{N}\rightarrow \mathbb{N}$ is computable. 
Formally in Type theory
$$\prod_{f : \mathbb{N}\rightarrow \mathbb{N}} \Big\lVert \sum_{e : \mathbb{N}} \prod_{x : \mathbb{N}} \sum_{z : \mathbb{N}} T(e,x,z) \times U(z) = f(x) \Big\rVert$$
where $T$ is Kleene's predicate, $e$ identifies a Turing machine, $x$ is some input to $f$ and its corresponding Turing machine $T_e$, finally $z$ is the computation history that $T_e$ goes through when given $x$ as the input, with $U(z)$ being the output at the end of the computation.
\end{defn}

In other words, Church's thesis assures us that for every $f : \mathbb{N} \rightarrow \mathbb{N}$ there exists some computable function that agrees with it on every input.


\chapter{LLPO}
In the following definition $s$ is, in places, taken to be an implicit argument. 
In the sections to follow, terms and types that use it implicitly, are supplied with explicit arguments when the need to be clear arises. 
\begin{defn}\label{LLPO}
The Lesser Limited Principle of Omniscience (LLPO), states that given binary sequence $s : \mathbb{N} \rightarrow \mathbf{2}$ and the fact that there is at most one occurence of $1$ in the sequence, formally 
$$\mathtt{atMost1one}\; s :\prod_{n_1 : \mathbb{N}} \prod_{n_2 : \mathbb{N}} s(n_1) = 1 \rightarrow s(n_2)= 1 \rightarrow n_1 = n_2$$
we can then have by the LLPO a witness for $\mathtt{p_{odd}} \vee \mathtt{p_{even}}$, where $\mathtt{p_{odd}}$ (with $s$ as an implicit argument) is the statement that for all odd positions $n$, $s(n) = 0$, formally $\mathtt{p_{odd}} \equiv \prod_{n : \mathbb{N}} (\mathtt{odd}(n) = 1) \rightarrow s(n) = 0$, and $\mathtt{odd} : \mathbb{N}\rightarrow \mathbf{2}$ with $\mathtt{odd}\; n = 1$ iff $n$ is odd. 
Similarly for $\mathtt{p_{even}}$. 
\end{defn}
LLPO can be viewed as a weaker form of the Law of Excluded Middle. 

\begin{lem} 
The Law of Excluded Middle implies the Lesser Limited Principle of Omniscience.
\end{lem}

\begin{proof}
By LEM we have a witness $ l_1 :\mathtt{p_{odd}} \vee \neg \mathtt{p_{odd}}$. 
Since this is a coproduct, by the relevant principle of induction, it's enough to prove LLPO from the disjuncts. 
\begin{itemize}
	\item $\mathtt{p_{odd}} \Rightarrow $ LLPO, trivially.

	\item $\neg \mathtt{p_{odd}}$, alongside LEM, implies that there exists odd $n_e : \mathbb{N}$ such that $s(n_e) = 1$. 
	We can now prove $\prod_{n : \mathbb{N}} \mathtt{even}(n) \rightarrow s(n) = 0$. 
	Let $n : \mathbb{N}$ such that $\mathtt{even}(n)$ is true. 
	By the definition of $s$, $s(n) = 0 \vee s(n) = 1$.
	We invoke the principle of induction of coproducts and prove LLPO from the disjuncts. 
	\begin{itemize}
		\item $s(n) = 0$, in this case we are done.
		\item $s(n) = 1$. 
		By $\mathtt{atMost1one}$ we have $n = n_e \Rightarrow $ $n$ even and odd which is a contradiction $c : \bot$. 
		Ex Falso,  $\mathtt{efq} : \bot \rightarrow s(n) = 0$. 
		Then $\mathtt{efq}(c) : s(n) = 0$.
	\end{itemize}
\end{itemize}
\end{proof}

\begin{lem}\label{llponeg}
If we replace the consequent of LLPO with its double negation, let's call it LLPO$^{\neg\neg}$, then we can prove it in Type Theory. 
\end{lem}
\begin{proof}
From LEM$\;\Rightarrow\;$LLPO we have LEM$\;\Rightarrow$LLPO$^{\neg\neg}$ by effectively the same proof. 
Since we invoke LEM only twice, we can propose the double negation of these two instances of LEM where the need arises, show that they are provable in our context and then drop the double negation since it's a modality and the goal is of the same modality. 
We effectively use the same method as when we drop truncations around hypotheses when proving mere propositions, only this time we are working with the double negation modality. 

The first instance we come across is $$\mathtt{p_{odd}} \vee \neg \mathtt{p_{odd}}$$
We want to prove $\neg \neg (\mathtt{p_{odd}} \vee \neg \mathtt{p_{odd}})$. 
Assume $q: (\mathtt{p_{odd}} \vee \neg \mathtt{p_{odd}}) \rightarrow \bot$. 
We compose $q$ with $|\cdot|$ to get 
$$q': (\mathtt{p_{odd}} + \neg \mathtt{p_{odd}}) \rightarrow \bot$$
We then have 
$$q' \circ \mathtt{inl} : \neg\mathtt{p_{odd}}$$
and
$$q' \circ \mathtt{inr} : \neg\neg \mathtt{p_{odd}}$$
which lead us to falsum.

The second instance is 
$$\Big\lVert\sum_{n_e : \mathtt{Odd}}s(n_e)=1 \Big\rVert \vee \neg \Big\lVert\sum_{n_e : \mathtt{Odd}}s(n_e)=1 \Big\rVert$$
Assume the negation of the above 
$$q : \neg \bigg(\Big\lVert\sum_{n_e : \mathtt{Odd}}s(n_e)=1 \Big\rVert \vee \neg \Big\lVert\sum_{n_e : \mathtt{Odd}}s(n_e)=1 \Big\rVert\bigg)$$
towards contradiction. 
We compose with $|\cdot|$ to rid ourselves of the truncation
$$q' : \neg \bigg(\Big\lVert\sum_{n_e : \mathtt{Odd}}s(n_e)=1 \Big\rVert + \neg \Big\lVert\sum_{n_e : \mathtt{Odd}}s(n_e)=1 \Big\rVert\bigg)$$
We then have
$$q' \circ \mathtt{inr} : \neg \neg \Big\lVert\sum_{n_e : \mathtt{Odd}}s(n_e)=1 \Big\rVert$$
and
$$q' \circ \mathtt{inl} : \neg \Big\lVert\sum_{n_e : \mathtt{Odd}}s(n_e)=1 \Big\rVert$$
Contradiction.
%We can leverage the fact that double negation is a modality, to construct a function that assumes the double negations of the LEM instances and concludes LLPO$^{\neg\neg}$.
\end{proof}

\begin{defn}
The Axiom of Countable Choice $(ACC)$ is adapted from its set-theoretic counterpart and states that if for every $n : \mathbb{N}$ there exists (merely) some $b : \lVert B\; n \rVert$, where $B : \mathbb{N} \rightarrow \mathcal{U}$ is a family of h-sets over $\mathbb{N}$, then there merely exists some $f : \prod_{n : \mathbb{N}} B\; n$.\\
More concisely
$$\Big(\prod_{n : \mathbb{N}} \big\lVert B\; n \big\rVert\Big) \rightarrow \Big\lVert \prod_{n : \mathbb{N}} B\; n \Big\rVert$$
\end{defn}

\begin{lem}\label{405}
Under Church's thesis, the following type is inhabited
$$\sum_{F : \mathbb{N} \rightarrow \mathbb{N} \rightarrow \mathbf{2}} \Big( \big(\prod_{m : \mathbb{N}} \mathtt{atMost1one}\; F(m) \big) \times \big( \prod_{f : \mathbb{N} \rightarrow \mathbf{2}} \big\lVert \sum_{m, k : \mathbb{N}} F(m, 2k+ f\; m) = 1 \big\rVert \big) \Big)$$
\end{lem}
\begin{proof}
First we provide a witness
$$G : \sum_{F : \mathbb{N} \rightarrow \mathbb{N} \rightarrow \mathbf{2} } \prod_{n : \mathbb{N}}\mathtt{atMost1one}\; F(n)$$
For $n, m : \mathbb{N}$ we pick the indexed $T_n$ Turing machine of our enumeration. 
We can decide whether $m$ is odd or even. 
\begin{itemize}
\item If it's odd, then there exists actual $k : \mathbb{N}$ such that $m = 2*k +1$. 
If $k$ is the Gödel number of the computation history that $T_n$ goes through when given $n$ as the input, furthermore, if it halts at the end of this computation with output $1$ then set $G.\mathtt{fst}\; n\; m :\equiv 1$. 
Otherwise set $G.\mathtt{fst}\; n\; m :\equiv 0$.
\item If it's even, then there exists actual $k$ such that $m = 2*k$. 
We work as in the odd case, with the only difference being that we set $G.\mathtt{fst}\; n\; m :\equiv 1$ iff the output of the halting computation is $0$.
\end{itemize}
Having defined $G.\mathtt{fst}$ it's easy to see that $G.\mathtt{snd}$ has a straightforward proof, which we ommit. 

We now define 
$$Q : (\mathbb{N} \rightarrow \mathbb{N} \rightarrow \mathbf{2}) \rightarrow \mathcal{U}$$ 
by 
$$Q\; F :\equiv \prod_{f : \mathbb{N} \rightarrow \mathbf{2}} \big\lVert \sum_{m, k : \mathbb{N}} F(m, 2k+ f\; m) = 1 \big\rVert$$
We want to prove $Q\; G.\mathtt{fst}$. 
Consider arbitrary $f : \mathbb{N} \rightarrow \mathbf{2}$. 
By CT we have that there exists, merely, $e : \mathbb{N}$ such that $T_e$ computes $f$, furthermore there exists $z : \mathbb{N}$ which $z$ is the Gödel number of the halting computation that $T_e$ goes through when given $e$ as an input and lastly $T_e$ halts at the end of this computation and outputs $j : \mathbb{N}$ where $j = f\; e$. 
Since it's decidable whether $j$ is $0$ or $1$, we can take cases on it.
In both cases we have $G.\mathtt{fst} (e, 2z + j) = 1$. 

We conclude the proof by providing $G.\mathtt{fst}$ as the first component and the product of $G.\mathtt{snd}$ with $Q\; G.\mathtt{fst}$ as the second.
\end{proof}

\begin{thm}
%In the context of HoTT, 
$$ACC \times CT \times LLPO \rightarrow \bot$$
\end{thm}
\begin{proof}
Let $G$ be the witness we reached in our proof of \cref{405}. 
Let $F :\equiv G.\mathtt{fst}$. 
By LLPO we can procure 
$$f : \prod_{n : \mathbb{N}} \lVert \mathtt{p_{odd}}(F\; n) + \mathtt{p_{even}}(F\;n) \rVert$$
By the ACC we get
$$f' : \big\lVert \prod_{n : \mathbb{N}}  \mathtt{p_{odd}}(F\; n) + \mathtt{p_{even}}(F\;n) \big\rVert$$
Since our goal is $\bot$, which is an h-prop, we can ignore the truncation and act as if we have access to
$$f'' : \prod_{n : \mathbb{N}}  \mathtt{p_{odd}}(F\; n) + \mathtt{p_{even}}(F\;n)$$
Let $e : \big( \mathtt{p_{odd}}(F\; n) + \mathtt{p_{even}}(F\;n)\big) \rightarrow \mathbf{2}$ be the function that sends $\mathtt{inl}\; \_ : \mathtt{p_{odd}}(F\; n) + \mathtt{p_{even}}(F\;n)$
to $1$ and $\mathtt{inr}\; \_$ to $0$. 
%By CT, the composition $e \circ f'' : \mathbb{N} \rightarrow \mathbf{2}$ has a $T_z$ that computes it. 
It's easy to see that the following holds
$$q : \prod_{n : \mathbb{N}} \big( (e\circ f''\; n = 1) \rightarrow \mathtt{p_{odd}}(F\; n) \big) \times \big( (e\circ f''\; n = 0) \rightarrow \mathtt{p_{even}}(F\; n) \big)$$
By $G.\mathtt{snd}$ we have that there exist $m,k : \mathbb{N}$ such that we have 
$$l : F(m, 2k + e\circ f''\; m) = 1$$
This should normally be truncated, but given the context, we are allowed to drop it. 

We then take cases on $(e\circ f''\; m = 0) + (e\circ f''\; m = 1)$. 
\begin{itemize}
	\item If $e\circ f''\; m = 0$ then by $q\; m$ we have $\mathtt{p_{even}}(F\; m)$. 
	This contradicts $l$. 
	\item Similarly, $e\circ f''\; m = 1$ is also in contradiction with $l$.
\end{itemize}
In either case we reach falsum, concluding the proof.
\end{proof}

\chapter{Extensional Type Theory}
There exists a model of extensional type theory that validates the following `Independence Principle'
$$\bigg( \prod_{s : \mathbf{2}^\mathbb{N}} P\; s \rightarrow \Big(\prod_{n : \mathbb{N}}s\; n = 0 \Big) \rightarrow \Big\lVert \sum_{z : \mathbb{N}} Q\; s\; z\Big\rVert \bigg)$$
$$\rightarrow \bigg( \prod_{s : \mathbf{2}^\mathbb{N}} P\; s \rightarrow  \Big\lVert \sum_{z:\mathbb{N}} \Big(\prod_{n : \mathbb{N}}s\; n = 0 \Big) \rightarrow Q \; s\;z\Big\rVert \bigg)$$
where $P: \mathbf{2}^\mathbb{N} \rightarrow \mathtt{Prop}$ and $Q : \mathbf{2}^\mathbb{N} \rightarrow \mathbb{N}\rightarrow \mathtt{Prop} $ are families of propositions. 
Putting it plainly, if the left hand side of the above is true, then $z : \mathbb{N}$ does not depend on the proof of the constantness of $s$ to $0$.

The same model also validates Markov's Principle
$$\Big(\prod_{n : \mathbb{N}}(P\; n + \neg P\; n)\Big) \rightarrow  \Big(\neg \prod_{n : \mathbb{N}}\neg P\; n\Big)\rightarrow \Big\lVert \sum_{n : \mathbb{N}} P\; n \Big\rVert$$
where $P : \mathbb{N} \rightarrow \mathtt{Prop}$ is a family of propositions over $\mathbb{N}$. 
Informally, Markov's Principle states that if we have a collection of decidable propositions and the fact that not all of them are false, then one of them must be true.

In this section we work in the context of this model of Extensional Type Theory.
\begin{defn}
Given $f : C \rightarrow D$ we say that its image is a proposition if for any $c_1,c_2 : C$, $f(c_1) = f(c_2)$.\\
Formally,
$$\mathtt{imageIsProp}\; f :\equiv \prod_{c_1,c_2 : C}f(c_1) = f(c_2)$$
\end{defn}
Let $A :\equiv \sum_{a: \mathbf{2}^\mathbb{N}}\mathtt{atMost1one}\;a$. 
When referring to elements of $A$ we implicitly mean the first part of the pair. 
Let
$$B : A \rightarrow \mathcal{U}$$
$$B :\equiv \lambda\; a.\; \mathtt{p_{odd}}\; a + \mathtt{p_{even}}\; a$$

We would like to have access to the following instance of IP in our intensional model as well.
\begin{multline*}
\prod_{a: A} \prod_{h : \sum_{k : B\;a \rightarrow \mathbb{N}}\mathtt{imageIsProp}(k)} \Bigg( 
\end{multline*}
\begin{multline*}
\bigg( \prod_{s : \mathbf{2}^\mathbb{N}} \Big(\big(\prod_{n : \mathbb{N}} s(n) = a(2 \cdot n)\big) \vee \big(\prod_{n : \mathbb{N}} s(n) = a(2\cdot n +1)\big) \Big) \rightarrow \\
	 \Big(\prod_{n : \mathbb{N}}s\; n = 0 \Big) \rightarrow \Big\lVert \sum_{z : \mathbb{N}}\prod_{p: B\; a}  k\; p = z \Big\rVert \bigg)
\end{multline*}
\begin{multline*}
\rightarrow \bigg( \prod_{s : \mathbf{2}^\mathbb{N}} \Big(\big(\prod_{n : \mathbb{N}} s(n) = a(2 \cdot n)\big) \vee \big(\prod_{n : \mathbb{N}} s(n) = a(2\cdot n +1)\big) \Big) \rightarrow \\
	  \Big\lVert \sum_{z : \mathbb{N}} \Big(\prod_{n : \mathbb{N}}s\; n = 0 \Big) \rightarrow \prod_{p : B\; a} k\; p = z \Big\rVert \bigg) \Bigg)
\end{multline*}
The fact that $\mathbb{N}$ is $\lVert B \rVert$-null, follows from this in Intensional Type Theory. 
Our plan of action shall be to find a consequent of it, IP', strong enough for our purposes, that has the right form so that by Theorem 6.1 of \cite{1905.03014} we have a model of intensional type theory that satisfies IP'. 

We drop the truncation around $\lVert \sum_{z : \mathbb{N}} \prod_{p : B\; a} k\; p = z \rVert$ and since this is the consequent of the antecedent, the resulting statement is weaker than the original IP. We then uncurry 4 times to reach what we propose as our IP', a function that takes four arguments in the form of a quaternary dependent pair

\begin{align*}
&a: &A
\\
&h: &\sum_{k : B\;a \rightarrow \mathbb{N}}\mathtt{imageIsProp}(k)
\\ &\_ :
\begin{split}
\bigg( \prod_{\bar{s} : \mathbf{2}^\mathbb{N}} \Big(\big(\prod_{n : \mathbb{N}} \bar{s}(n) = a(2 \cdot n)\big) + \big(\prod_{n : \mathbb{N}} \bar{s}(n) = a(2\cdot n +1)\big) \Big) \rightarrow \\
	\quad \Big(\prod_{n : \mathbb{N}}\bar{s}\; n = 0 \Big) \rightarrow  \sum_{\bar{z} : \mathbb{N}} \prod_{p : B\; a} k\; p = \bar{z}  \bigg)
\end{split}
\\
&r : & \sum_{s : \mathbf{2}^\mathbb{N}} \Big(\big(\prod_{n : \mathbb{N}} s(n) = a(2 \cdot n)\big) + \big(\prod_{n : \mathbb{N}} s(n) = a(2\cdot n +1)\big) \Big)
\end{align*}
and has return type
$$\Big\lVert \sum_{z : \mathbb{N}}\Big(\prod_{n : \mathbb{N}}s\; n = 0 \Big) \rightarrow \prod_{p : B\; a} k\; p = z \Big\rVert$$
The observant reader should have noticed that the disjunctions in the third and fourth arguments have been replaced with coproducts. 
This seemingly makes the unnamed third argument weaker which would have the undesirable effect of potentially strengthening IP' beyond IP. 
Luckily the consequent of $\_$ is a mere proposition which means that swapping disjunction for coproduct does not actually change the strength of the argument. 
In the case of $r$ the argument becomes stronger, which is satisfactory in itself. 
\begin{lem}
Consider families of types $B_1, B_2$ where $B_i : A_i \rightarrow \mathcal{U}$ and $A_i : \mathcal{U}$, $i \in \{1,2\}$. 

Define $B : A_1 +A_2 \rightarrow \mathcal{U}$, $B(\mathtt{inl}\; a) :\equiv  B_1(a)$ for $a : A_1$ and $B(\mathtt{inr}\; a) :\equiv  B_2(a)$ for $a : A_2$. 
Then the following type
$$\prod_{a : A_1}\lVert B_1\; a\rVert \times \prod_{a : A_2}\lVert B_2\;a \rVert$$
is equivalent to
$$\prod_{a : A_1 + A_2} \lVert B\; a \rVert$$
\end{lem}
\begin{proof}
$(\Rightarrow )$ Suppose that we have $p : \prod_{a : A_1}\lVert B_1\; a\rVert \times \prod_{a : A_2}\lVert B_2\;a \rVert$. 
We want to construct a function $\prod_{a : A_1 + A_2} \lVert B\; a \rVert$. 
We use induction on the coproduct and wlog work out only the case $\prod_{a : A_1} \lVert B(\mathtt{inl}\; a)\rVert$. 
For $a : A_1$ we have that $p.\mathtt{fst}(a)$ is a witness of $\lVert B(\mathtt{inl}\; a) \rVert$.

$(\Leftarrow) $ Suppose $q : \prod_{a : A_1 + A_2} \lVert B\; a \rVert$. 
We want to construct a witness for $\prod_{a : A_1}\lVert B_1\; a\rVert \times \prod_{a : A_2}\lVert B_2\;a \rVert$. 
We do so by constructing witnesses for both of the constituents of the product. 
Wlog we do that only for $\prod_{a : A_1}\lVert B_1\; a\rVert$. 
Let $a : A_1$. Then $q(\mathtt{inl}\; a) : \lVert B_1\; a \rVert$, by the definition of $B$.
\end{proof}
The above lemma generalizes from the case of two families of types $B_1, B_2$ to a finite collection of families $B_1,\ldots,B_n$. 
This fact enables us to reformulate a finite collection of statements, each one in the correct form, to a single statement in that same form which is equivalent to their conjunction. 

We are now in the position to assume that we have access to a model $\mathcal{E}$ of univalent type theory, in which CT, MP and IP' hold. 

\section{Null types in E}
The proofs in this section are all about $\lVert B \rVert$-nullness i.e. equivalence between function types (see Definition \ref{Bnull}). 
Our approach in showing that some $F: C\rightarrow D$ is a map with contractible fibers, is to provide for any $d :D$ some $c :C$ such that $f(c) = d$ and then show that such a $c$ is unique, where $C,D$ are said function types. 
\begin{lem}\label{NisBnull}
$\mathbb{N}$ is $\lVert B \rVert$-null in $\mathcal{E}$.
\end{lem}
\begin{proof}
Given $a : A$ and $f : \lVert B\; a \rVert \rightarrow \mathbb{N}$ we need to prove that there exists unique $f' : \mathbf{1} \rightarrow \mathbb{N}$ through which $f$ factors, in the sense that $f = f' \circ g$, where $g : \lVert B\; a \rVert \rightarrow \mathbf{1}$ is the sole inhabitant of its function space. 
Functions with $\mathbf{1}$ as their domain are constant. 
We therefore need to find some $z : \mathbb{N}$ so that $\lambda\; \_.\; z$ is $f'$. 
To that end, we will use IP' which we've proven true in $\mathcal{E}$. 
We invoke it twice. 
In both times, the first three arguments shall be the same. 
The choice for the first argument, $a$, is evident. 

We provide the composition $f \circ |\cdot| : B\; a \rightarrow \mathbb{N}$ as $k$ of second argument. 
We need to provide a witness for the second part of $h$, $\mathtt{imageIsProp(f \circ |\cdot|)}$. 
Let $q_1,q_2 : B\;a$. 
We need to show that $f\;|q_1| = f\; |q_2|$. 
Since $\lVert B\; a\rVert$ is a proposition, we have $|q_1| = |q_2|$ and then by action on paths we get the desired equality.

We need to construct a witness for the third argument. 
Let $\bar{s} : \mathbf{2}^\mathbb{N}$ such that $\bar{s}$ is constantly equal to $0$ and at the same time it is \textit{actually} equal to the odd subsequence of $a$ or to the even one i.e. $\big(\prod_{n : \mathbb{N}} \bar{s}(n) = a(2 \cdot n)\big) + \big(\prod_{n : \mathbb{N}} \bar{s}(n) = a(2\cdot n +1)\big)$. 
We take cases on this coproduct. 
Wlog suppose that $\prod_{n : \mathbb{N}} \bar{s}(n) = a(2 \cdot n)$. 
We can use this to construct a witness $\mathtt{evenSubseqIsZero} : B\;a$. 
We then put forward $f\;|\mathtt{evenSubseqIsZero}|$ as $\bar{z}$. 
Recall that we've set $k$ to be $f\circ |\cdot|$.
We need to show that for arbitrary $p : B\;a $, we have $f\; |p| = f\;|\mathtt{evenSubseqIsZero}|$. 
This follows from $\lVert B\;a\rVert$ being a proposition and action on paths. 

Finally, for argument $r$ we provide the odd $s_1$ and even $s_2$ subsequences of $a$, along with proofs that they are indeed subsequences of $a$ and we get hold of $z_1 : \mathbb{N}$ and $z_2 : \mathbb{N} $ respectively, along with $\zeta_i : \prod_{p : B\; a}\big(\prod_{n : \mathbb{N}}s_i\; n = 0 \big)\rightarrow\big(f\circ |\cdot|(p)=z_i\big)$ for $i \in \{1,2\}$. 
Recall that the current goal is a mere proposition, namely that merely exists some $f'$, that is why we can act as if we have actual $z_1$ and $z_2$. 
Equality on $\mathbb{N}$ is decidable, therefore $(z_1 = z_2) + (z_1 \neq z_2)$ is provable. 

We will first prove that the existence of a candidate for $f'$ follows from both constituents of the coproduct.
\begin{itemize}
\item First the case where $z_1= z_2$. 
We arbitrarily pick $z_1$ and propose $f' : \equiv \lambda \;\_.\; z_1 : \mathbf{1} \rightarrow \mathbb{N}$. 
By function extensionality we reduce proving $f= f' \circ g$ to proving $f\; c = f'(g\; c)$ for arbitrary $c : \lVert B\; a\rVert$.\\
Since our goal is a mere proposition, we can act as if we have access to an actual $b : B\; a$. 
We take cases on $b$. 
\begin{itemize}
\item In the first case we have $\mathtt{p_{odd}}$ which trivially leads us to $b_1 : \prod_{n : \mathtt{N}}s_1 = 0$. 
We then have $\zeta_1(b)(b_1) : f(|b|) = z_1$. 
Since $\lVert B\; a\rVert$ is a mere proposition, we have $|b| =c$. 
By action on paths on this and $f$ we get $f\; c= z_1$. 
By definition, $f'(g(c)) = z_1$, which we concatenate with $f(c) = z_1$ to conclude get $f\; c = f'(g\; c)$ and this case.
\item In the case where $\mathtt{p_{even}}$, we similarly construct $b_2 : \prod_{n : \mathtt{N}}s_2 = 0$. 
We then have $\zeta_2(b)(b_2) : f(|b|) = z_2$. 
The proof follows closely the previous case, only this time around we have to include $z_1=z_2$ in the concatenation of paths.

\end{itemize}
% We then apply $\zeta_1$ to it to get $\zeta_1\; b : f\circ |\cdot|(b)=z_1$ from which we deduce $f(|b|) = z_1$. 
% Since $\lVert B\; a\rVert$ is a mere proposition, we have $|b| =c$. 
% By action on paths on this and $f$ we get $f\; c= z_1$. 
% By definition, $f'\circ g(c) = z_1$, which we concatenate with $f(|b|) = z_1$ to conclude this case.

% We then take cases on $b \equiv \mathtt{p_{odd}}(a) + \mathtt{p_{even}}(a)$.  
% \begin{itemize}
% \item We first consider the case $q : \mathtt{p_{odd}}\; a$. 
% Through this we can still get access to some actual $b' : B\; a$ with $b' :\equiv \mathtt{inl}(q)$. 
% The odd subsequence is constantly $0$, which we can provide to argument $r$ of IP' alongside with $b'$ (and $s_1$ as the first argument) to deduce $f \circ \lVert\cdot\rVert(b') = z_1$. 
% Using this we then prove $f\; c = z_1$. 
% We concatenate this with $f' \circ g \; c = z_1$ to conclude this case. 
% \item The $\mathtt{p_{even}}$ case is very similar. 
% The only difference is that we provide $s_2$ as the first argument to the rhs of the IP and then have $z_1=z_2$ be part of the concatenation of paths when we prove homotopy between $f$ and $f' \circ g$. 
% \end{itemize}

\item Now consider the case $z_1 \neq z_2$. 
For any $n : \mathbb{N}$, $a\; n = 1$ is decidable. 
Suppose that $\prod_{n : 
\mathbb{N}} a\; n \neq 1$. 
Then $a$ and by extension $s_1$ and $s_2$ are constantly $0$. 
We use these facts to construct $b : B\;a$ and $b_i : \prod_{n : \mathtt{N}}s_i = 0$ and then concatenate $\zeta_i(b)(b_i)$ like before to get $z_1 = f\; b = z_2$. 
This contradicts $z_1 \neq z_2$. 
We have proven $\neg \prod_{n : 
\mathbb{N}} a\; n \neq 1$. 
By Markov's Principle there exists $n_1 : \mathbb{N}$ such that $a\; n_1 = 1$. 
We pick the subsequence $s_0$ with parity opposite to $n_1$ and let $f'$ be always equal to the corresponding $z_0$. 
We prove $\prod_{n : \mathbb{N}} s_0\; n = 0$ by induction on $\mathbb{N}$ and cases on $(s_0 \;n = 0) + (s_0\;n = 1)$. 
In the case where $s_0\; n = 1$ we can reach falsum because $n$ will have to be both even and odd (in $a$) by $\mathtt{atMost1one} \; a$. Ex falso trivializes this case. 
Now that we have established this too, we can construct $b : B\; a$ and have enough arguments for IP' to output the desired equality $f\; b = z_0$ which proves homotopy between $f$ and $(\lambda\;\_.\; z_0) \circ g$.
\end{itemize}
We still need to prove the uniqueness of $f'$. 
Since $\mathbf{1}$ is finite and equality on $\mathbb{N}$ is decidable, we can decide equality on the function space 
$$p : \prod_{f_1,f_2 : \mathbf{1}\rightarrow \mathbb{N}} f_1 = f_2 + f_1 \neq f_2$$
We will use this to prove that $\sum_{h : \mathbf{1}\rightarrow \mathbb{N}} f= h \circ g$ is a mere proposition, which entails the uniqueness of $f'$. 
Consider $h_1,h_2 : \mathbf{1}\rightarrow \mathbb{N}$ such that $f= h_i \circ g,\; i \in \{1,2\}$. 
By $p$ we have $(h_1 = h_2) + (h_1 \neq h_2)$. 
We use induction on the coproduct. 
The left case is trivial. 
For the right case we have $q : h_1 \neq h_2$ and we will first try to prove $B\; a \rightarrow \bot$ as a stepping stone. 
Consider $b : B\; a$. 
Since $h_1 \circ g = f = h_2 \circ g$, by homotopy $h_1 (g\; b) = h_2 (g\; b)$. 
We can then prove $\prod_{*: \mathbf{1}} h_1 (*) = h_2 (*)$. 
By function extensionality $h_1 = h_2$. 
But this contradicts $q$, so we reach $\bot$. 
We've just proven $\tau : B\;a \rightarrow \bot$. 
We still have to prove $h_1 = h_2$, the main goal. 
By \cref{llponeg} we have $(B\; a \rightarrow \bot) \rightarrow \bot$ and by $\tau$ we get $\bot$. 
We conclude the proof by Ex Falso.
\end{proof}

\begin{cor}
Similarly, $\mathbf{2}$ is $\lVert B \rVert$-null in $\mathcal{E}$. 
\end{cor}

\begin{lem}
In $\mathcal{E}$ if $C, D$ are $\lVert B \rVert$-null then so is $C + D$.
\end{lem}
\begin{proof}
We need to show that for $f : \lVert B\; a\rVert \rightarrow C+D$ there exists unique $f' : \mathbf{1} \rightarrow C+D$ such that $f = f' \circ e$, where $e : \lVert B\; a\rVert \rightarrow \mathbf{1}$ is the sole inhabitant of its function space. 
We define $$g : C + D \rightarrow \mathbf{2}$$ 
using components
$$g_C :\equiv \lambda\; \_.\; 0 : C \rightarrow \mathbf{2}$$
$$g_D :\equiv \lambda\; \_.\; 1 : D \rightarrow \mathbf{2}$$
Recall that in $\mathcal{E}$ we have access to a witness of IP' which takes the following arguments in the form of a 4-product
\begin{align*}
&a: &A
\\
&\bar{f}: &B\; a \rightarrow C+D
\\ &\_ :
\begin{split}
\bigg( \prod_{\bar{s} : \mathbf{2}^\mathbb{N}} \Big(\big(\prod_{n : \mathbb{N}} \bar{s}(n) = a(2 \cdot n)\big) \vee \big(\prod_{n : \mathbb{N}} \bar{s}(n) = a(2\cdot n +1)\big) \Big) \rightarrow \\
	\quad \Big(\prod_{n : \mathbb{N}}\bar{s}\; n = 0 \Big) \rightarrow  \sum_{z : \mathbf{2}} \prod_{b : B\; a} g(\bar{f}\; b) = z  \bigg)
\end{split}
\\
&r : & \sum_{s : \mathbf{2}^\mathbb{N}} \Big(\big(\prod_{n : \mathbb{N}} s(n) = a(2 \cdot n)\big) \vee \big(\prod_{n : \mathbb{N}} s(n) = a(2\cdot n +1)\big)
\end{align*}
and returns
$$\Big\lVert \sum_{z : \mathbf{2}}\Big(\prod_{n : \mathbb{N}}s\; n = 0 \Big) \rightarrow \prod_{b : B\; a} g(\bar{f}\; b) = z \Big\rVert$$
We provide this with the arguments required, where $\bar{f}$ is $f$ composed with the truncation map $|\cdot| : B\; a \rightarrow \lVert B\; a\rVert$, to get a witness 
$$\bar{z} : \sum_{z : \mathbf{2}} \prod_{b : B\; a} g(\bar{f}\; b) = z$$ 
Note that we dropped the truncation since our goal is a mere proposition, therefore we can act as if we have an actual $\bar{z}$. 
Using $\bar{z}$ we can prove that
$$y : \sum_{z : \mathbf{2}} \prod_{b : \lVert B\; a\rVert} g(f\; b) = z$$
We can prove that 
$$y= 0 + y = 1$$ 
We use induction on this coproduct to prove our goal. 
Without loss of generality, we argue only for the case of $y = 0$. 
As a first step towards providing a candidate for $f'$, we will try to construct some $h : \lVert B\; a\rVert \rightarrow C$ such that $\mathtt{inl} \circ h = f$. 
\begin{center}
\begin{tikzcd}
                                                             & \mathbf{1} \arrow[rdd, "f'", dashed, bend left] \arrow[d, "h'"', dashed]              &                    &            \\
                                                             & C \arrow[rd, "\mathtt{inl}"]                                                          &                    &            \\
B\; a \arrow[rr, "\bar{f}"', bend right=49] \arrow[r, "|*|"] & \lVert B\;a\rVert \arrow[r, "f"] \arrow[u, "h", dashed] \arrow[uu, "e", bend left=49] & C+D \arrow[r, "g"] & \mathbf{2}
\end{tikzcd}
\end{center}
Let $b : \lVert B\; a\rVert$. 
First, note that we can prove 
$$\zeta : \prod_{a: C+D} \Big(\big(\sum_{c : C} \mathtt{inl}\; c = a\big) + \big( \sum_{d:D} \mathtt{inr}\; d = a \big) \Big)$$
by induction on $C+D$. 
We then take cases on 
$$\zeta (f\; b) : \big(\sum_{c : C} \mathtt{inl}\; c = f\; b\big) + \big( \sum_{d:D} \mathtt{inr}\; d = f\; b \big)$$
\begin{itemize}
	\item If $\bar{c} : \sum_{c : C} \mathtt{inl}\; c = f\; b$, we define $h\; b :\equiv \bar{c}$.
	Clearly $\mathtt{inl} \circ h = f$.
	\item If $\sum_{d : D} \mathtt{inl}\; d = f\; b$, then $g(f\;b) = 1$ but since $y=0$ we also have $g(f\;b) = 0$. Contradiction. 
	This case is resolved by Ex Falso.
\end{itemize}
Since $C$ is $\lVert B \rVert$-null, there exists $h' : \mathbf{1} \rightarrow C$ such that $h = h' \circ e$. 
Observe that $\mathtt{inl} \circ h' : \mathbf{1} \rightarrow C+D$, furthermore $(\mathtt{inl} \circ h') \circ e= f$. 
We have found a valid candidate for $f'$, what is left is to show it's unique. 
% We reduce this to showing that $\sum_{f' : \mathbf{1} \rightarrow C+D} f' \circ e = f$ is a mere proposition. 

% Let $f_1', f_2' : \sum_{f' : \mathbf{1} \rightarrow C+D} f' \circ e = f$. 
% We can prove that for $* : \mathbf{1}$, we have 
% $$\big(\sum_{c: C} f_i'\;* = \mathtt{inl}\; c\big) + \big(\sum_{d: D} f_i'\;* = \mathtt{inr}\; d\big)$$
% for $i \in \{ 0,1 \}$. 
% We then distinguish cases:
% \begin{itemize}
% \item If $c_1 : \sum_{c: C} f_1'\;* = \mathtt{inl}\; c$ and $c_2 : \sum_{c: C} f_2'\;* = \mathtt{inl}\; c$, 
% then 
% \end{itemize}
Consider any $f'$ candidate. 
We can construct a $f'_C : \mathbf{1} \rightarrow C$ so that $f'$ factors through it, $f' = \mathtt{inl} \circ f'_C$, with our construction being similar to that of $h$ earlier. 
We want to show that $f' = \mathtt{inl} \circ h'$ or $\mathtt{inl} \circ f'_C = \mathtt{inl} \circ h'$. 
We will show that $h' = f'_C$. 
Since $C$ is $\lVert B \rVert$-null, $\sum_{k : \mathbf{1} \rightarrow C} k \circ e = h$ is a mere proposition. 
Therefore showing $h' = f'_C$ is reduced to showing $f'_C \circ e = h$. 
We use function extensionality. 
We know that for arbitrary $b : \lVert B\; a \rVert$, $\mathtt{inl} (f'_C \circ e\; b ) = \mathtt{inl} (h\; b)$. 
By 2.12.1 of \cite{hottbook} we have $(\mathtt{inl}\; a_1 = \mathtt{inl}\; a_2) \simeq (a_1 = a_2)$. 
We can then deduce that $f'_C\circ e\; b = h\; b$. This concludes the proof of uniqueness of $h'$.

\end{proof}

\begin{lem}
In $\mathcal{E}$ if $C$ is $\lVert B \rVert$-null then for any $c_1, c_2 : C$, the identity type $c_1=_C c_2 : \mathcal{U}$ is $\lVert B\rVert$-null. 
\end{lem}
\begin{proof}
Follows immediately from the fact that $\mathcal{L}_{\lVert B \rVert}$ is a modality and the fourth datum of Definition \ref{modality_definition}.
\end{proof}

\section{CT + LLPO + HoTT}

We form a null types model $\mathcal{E}_{\lVert B \rVert}$ as defined in Section 5 of \cite{1905.03014} based on $\mathcal{E}$ from before. 
The types of this model are the $\lVert B \rVert $-null types of $\mathcal{E}$. 
By \cref{NisBnull} we have that $\llbracket \mathbb{N} \rrbracket^{\mathcal{E}_{\lVert B \rVert}} = \llbracket \mathbb{N} \rrbracket^\mathcal{E}$. 
Let $\mathcal{L}_{\lVert B \rVert} : \mathcal{U} \rightarrow \mathcal{U}$ be the $\lVert B \rVert$-nullification operator on $\mathcal{E}$. 
On $\mathcal{E}_{\lVert B \rVert}$, dependend products, dependent sums and identity on naturals are all inherited from $\mathcal{E}$, while propositional truncation $\llbracket \lVert \cdot \rVert \rrbracket^{\mathcal{E}_{\lVert B \rVert}}$ is $\mathcal{L}_{\lVert B \rVert} \lVert\cdot \rVert $ on $\mathcal{E}$.

We can then deduce that the interpretation of CT in $\mathcal{E}_{\lVert B \rVert}$ is the type 
$$\prod_{f : \mathbb{N}\rightarrow \mathbb{N}} \mathcal{L}_{\lVert B \rVert}\;\Big\lVert \sum_{e : \mathbb{N}} \prod_{x : \mathbb{N}} \sum_{z : \mathbb{N}} T(e,x,z) \times U(z) = f(x) \Big\rVert$$
in $\mathcal{E}$. To clarify, the types $T$ and $U$ remain the same because they are functions from naturals to naturals and we know that $\mathbb{N}$ is the same in both $\mathcal{E}_{\lVert B \rVert}$ and $\mathcal{E}$. 
It's trivial to construct an inhabitant, in $\mathcal{E}$, of the above type using a witness of CT, which we know to be true in $\mathcal{E}$. 

We shall now have a look at the interpretation of LLPO in $\mathcal{E}_{\lVert B \rVert}$. 
Recall the full definition \ref{LLPO} of LLPO
$$\prod_{a : A}\lVert \mathtt{p_{odd}}(a.\mathtt{fst}) + \mathtt{p_{even}}(a.\mathtt{fst}) \rVert$$
where $$A :\equiv \sum_{a: \mathbf{2}^\mathbb{N}}\mathtt{atMost1one}\;a$$
Since $\mathtt{atMost1one}$ is formed using $\mathbb{N}$, $\mathbf{2}$, equality on naturals and $\mathbf{2}$, dependent sum and product, all of which are preserved under modalities, we have that the interpretation of $\mathtt{atMost1one}$ in $\mathcal{E}_{\lVert B \rVert}$ is the same type as in $\mathcal{E}$. 
Similarly the interpretations of $A$, $\mathtt{p_{odd}}$ and $\mathtt{p_{even}}$ are the same. 
Therefore, the interpretation of LLPO in $\mathcal{E}_{\lVert B \rVert}$ is the type 
$$\prod_{a : A}\mathcal{L}_{\lVert B \rVert}\;\lVert \mathtt{p_{odd}}(a.\mathtt{fst}) + \mathtt{p_{even}}(a.\mathtt{fst}) \rVert$$
in $\mathcal{E}$. 
We would like to have a witness of this type. 
Let $a : A$. 
We once again utilize our shorthand $B\; a \equiv \mathtt{p_{odd}}(a.\mathtt{fst}) + \mathtt{p_{even}}(a.\mathtt{fst})$. 
Note that $\mathcal{L}_{\lVert B \rVert}\;\lVert B\; a \rVert$ is $\lVert B \rVert$-null, which means that we have the following equivalence
$$\mathcal{L}_{\lVert B \rVert}\; \lVert B\;a\rVert \simeq \lVert B\; a \rVert \rightarrow \mathcal{L}_{\lVert B \rVert}\;\lVert B\; a \rVert$$
So it is enough to prove $\lVert B\; a \rVert \rightarrow \mathcal{L}_{\lVert B \rVert}\;\lVert B\; a \rVert$, for which we already have a witness in the form of the first datum of Definition \ref{modality_definition}, since $\mathcal{L}_{\lVert B \rVert}$ is a modality. 

Therefore, $\mathcal{E}_{\lVert B \rVert}$ is a model of univalent type theory where both CT and LLPO hold.

\printbibliography

\end{document}
